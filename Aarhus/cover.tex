%%%%%%%%%%%%%%%%%%%%%%%%%%%%%%%%%%%%%%%%%
% Plain Cover Letter
% LaTeX Template
% Version 1.0 (28/5/13)
%
% This template has been downloaded from:
% http://www.LaTeXTemplates.com
%
% Original author:
% Rensselaer Polytechnic Institute 
% http://www.rpi.edu/dept/arc/training/latex/resumes/
%
% License:
% CC BY-NC-SA 3.0 (http://creativecommons.org/licenses/by-nc-sa/3.0/)
%
%%%%%%%%%%%%%%%%%%%%%%%%%%%%%%%%%%%%%%%%%

%----------------------------------------------------------------------------------------
%	PACKAGES AND OTHER DOCUMENT CONFIGURATIONS
%----------------------------------------------------------------------------------------
\documentclass[10pt]{letter} % Default font size of the document, change to 10pt to fit more text
%\usepackage[nochapters]{classicthesis} % Use the classicthesis style for the style of the document
\usepackage[LabelsAligned]{currvita} % Use the currvita style for the layout of the document
\usepackage[]{geometry}

%\usepackage{newcent} % Default font is the New Century Schoolbook PostScript font 
%\usepackage{helvet} % Uncomment this (while commenting the above line) to use the Helvetica font

% Bibliography
\makeatletter
\newenvironment{thebibliography}[1]
     {\list{\@biblabel{\@arabic\c@enumiv}}%
           {\settowidth\labelwidth{\@biblabel{#1}}%
            \leftmargin\labelwidth
            \advance\leftmargin\labelsep
            \usecounter{enumiv}%
            \let\p@enumiv\@empty
            \renewcommand\theenumiv{\@arabic\c@enumiv}}%
      \sloppy
      \clubpenalty4000
      \@clubpenalty \clubpenalty
      \widowpenalty4000%
      \sfcode`\.\@m}
     {\def\@noitemerr
       {\@latex@warning{Empty `thebibliography' environment}}%
      \endlist}
\newcommand\newblock{\hskip .11em\@plus.33em\@minus.07em}
\makeatother

\pagestyle{empty}
% Margins
\topmargin=-1in % Moves the top of the document 1 inch above the default
\textheight=9.5in % Total height of the text on the page before text goes on to the next page, this can be increased in a longer letter
\oddsidemargin=-10pt % Position of the left margin, can be negative or positive if you want more or less room
\textwidth=6.5in % Total width of the text, increase this if the left margin was decreased and vice-versa

\let\raggedleft\raggedright % Pushes the date (at the top) to the left, comment this line to have the date on the right

%\address{test}

\begin{document}

%----------------------------------------------------------------------------------------
%	ADDRESSEE SECTION
%----------------------------------------------------------------------------------------
\begin{letter}{}%{Department  \\ University of X\\ City, State Area Code}
%----------------------------------------------------------------------------------------
%	YOUR NAME & ADDRESS SECTION
%----------------------------------------------------------------------------------------
\begin{center}
	\textbf{Philipp Jan Andries Stassen} \\ % Your name
%\vspace{20pt} \hrule height 1pt % If you would like a horizontal line separating the name from the address, uncomment the line to the left of this text
Erik Sandbergs Gata 46 \\ 169 34 Solna, Sweden \\ +49 151 4242 9379 \\ philipp.stassen@gmail.com % Your address and phone number
\end{center} 

\vspace{1.5em}
\begin{center} \large\bf PhD student for Center for Basic Research in Program Verification
\end{center}
\bigskip
 % Your name for the signature at the bottom

%----------------------------------------------------------------------------------------
%	LETTER CONTENT SECTION
%----------------------------------------------------------------------------------------

Dear Sir or Madam,

\smallskip

to thrive as a researcher in mathematical logic and computer science, I herewith apply as a PhD student in the Center for Basic Research in Program Verification. 
I am capable of conducting the relevant research in this area and am excited to develop new reasoning techniques in the field of type theory and category theory.
%Großes analytisches Denken und arbeite mich schnell in komplexe Sachverhalte ein. Diverses Wissen
%Because of my experience in the area of type theories and mathematical logic as well as my general mathematical background I believe that 
Both my quick comprehension of complex topics as well as my passion for mathematical logic make me the most suitable candidate for a PhD under the supervision of Lars Birkedal.


%I'm 26 years old. 
%I am a student at the Stockholm University and by concluding my master thesis "An analysis of Curien's explicit syntax for Dependent Type Theory" on the 30th of March 2020 I acquired the Master of Science in Mathematics. 
%I have the German and Dutch citizenship, but I grew up in the Frankfurt area where I acquired the higher education entrance qualification. In 2014 I started studying mathematics at the Goethe Universität Frankfurt and then moved to the Rheinische Friedrich-Wilhelms-University in Bonn in 2015. 
%During my time in Bonn I acquired knowledge in mathematical logic by attending courses in classical Set Theory, Model Theory and Forcing and also a seminar on Type Theory. I decided to continue the seminar with a Bachelor thesis dealing more specifically with homotopy type theory.
%I finished my Bachelor thesis "Formalism of homotopy type theory and its application to logic" in January 2018 in Bonn. 


My desire to focus on theoretical computer science and logic was sparked in May 2018, when I attended the summer school of the Hausdorff trimester program ``Types Sets and Constructions''. 
Studying at Stockholm university provided me with the perfect environment to focus more specifically on the constructive aspects of mathematical logic and theoretical computer science.
I completed courses in Type Theory, Modal and Temporal Logic, Theory for Computation and Formal Language. 
Additionally, I attended the PhD-courses ``Proof Theory and Subsystems of Second Order Arithmetic" and ``Infinity-categories" during my masters studies.

%I completed my master thesis together with my supervisors Peter LeFanu Lumsdaine and Guillaume Brunerie at the 30st of March 2020 with the title "An analysis of Curiens explicit syntax for Dependent Type Theory". 

%I was fortunate that I could attend the conferences "Proof, Computation, Complexity" and "Mathematical Logic and Constructivity" in the summer 2019 and by that learn about the latest discussions and developments in constructive mathematics and in particular univalent foundations of mathematics.
%Therefore, the PhD position would be an excellent chance to further work and learn about . 


%I am confident that I do not only possess the knowledge to to research in this area, but that I can contribute to the study of categorial semantics of type theory by following up on my master thesis. 

I completed my master thesis together with my supervisors Peter LeFanu Lumsdaine and Guillaume Brunerie on the 30st of March 2020 with the title ``An analysis of Curiens explicit syntax for Dependent Type Theory".
We study a coherence problem that arises when interpreting dependent type theories in \emph{locally cartesian closed categories} (as in \cite{seely}), which causes the interpretation to not be well-defined. 
Pierre-Louis Curien suggested in \cite{curien} to resolve said problem by designing an intermediate type theory which possesses an additional term constructor (the \emph{explicit coercion}). For this syntax, the interpretation will turn out to be well defined.
As also discussed in \cite{lumsdainehomotopy}, relating the two syntaxes would give rise to a new understanding of categorial semantics.
%To resolve that problem we design an intermediate type theory that has an additional term constructor, the \emph{explicit coercion}. 
%Pierre-Louis Curien has proven in \cite{curien} that locally cartesian closed categories provide strict models for this new syntax – thereby resolving the mismatch by modifying the syntax instead. 
%It would be interesting to know how these models relate to the original, unmodified type theory. 
%Therefore, we try to  in what sense the syntaxes are equivalent.
	My thesis proves a strong equivalence theorem. 
	Ultimately, this result functions as a tool to construct \emph{weak models} for dependent type theories. 
	The definitions and proofs are formalized in the proof assistant Agda\footnote{Please find the current status at https://github.com/philippstassen/initiality/tree/develop1}.

	For your future research I can benefit you through my deep insight in theoretical computer science and my vast experience in the field of type theory. 
	%With my theoretical insight in computer science and proof assistants I want to benefit the research and teaching of your institute.

%	My goal is to proceed formalizing the results in Agda or Coq. 
%	By enriching the very slim type theory with new constructors and adapting the equality rules appropriately, I want to extend the results to extensional type theories, intensional type theories or even homotopy type theories, as already suggested in \cite{curien-garner-hofmann}.
%	In a second step I want to elaborate on the semantical consequences, that also Peter LeFanu Lumsdaine and Chris Kapulkin already discuss in \cite{lumsdainehomotopy}. 

	I am determined to acquire techniques and skills that are practically relevant inside and outside the field of mathematical logic – and make the results available for a broader audience.
	%fields inside and outside of mathematics - and make the theoretic results available for a broader audience.
	With my experience in and excitement for computer science, 
%	as well as my diversified thinking I will contribute to the discussions and work of the mathematical logic group as well as transfer that knowledge also to other areas inside and outside of mathematics.
%I would be delighted to collaborate with computer science departments: An already ``popular'' application for proof assistants like Coq and Agda is program verification.
%But I would also love to develop and explore more general reasoning techniques that possibly take modalities or quantitative systems into consideration (such as studied in the philosophy department of SU and the mathematical statistic department).
%I believe that being admitted to 
your PhD position provides me with the perfect opportunity to work interdisciplinary and to create practical applications for type theory.


Sincerely yours,

\smallskip
Philipp Stassen

\vfill
\encl{Curriculum vitae, Project description} % List your enclosed documents here, comment this out to get rid of the "encl:"

%----------------------------------------------------------------------------------------
\bibliographystyle{unsrt}
\bibliography{bibliography}

\end{letter}
\end{document}

